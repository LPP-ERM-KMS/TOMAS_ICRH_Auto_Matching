\chapter{Overview}
\section{Introduction}
This work aims to be a succesful implementation of an automatic matching
algorithm for the Ion Cyclotron Resonance Heating (ICRH) antennna, a circuit
drawing of this matching system is shown in figure \ref{fig:Circuit}.  The antennae are
facing a plasma with ever changing conditions: the density, temperature and
even species (He/H/Ar) may vary; As such the capacitance of the various capacitors
have to be varied every time the conditions change as to maximize power transfer.\\

At the time of writing this is carried out manually by first lowering the
current going to the antenna by adjusting $C_a$, then varying capacitances of $C_p$ and
$C_s$, doing a frequency sweep and looking where the power transferred is
maximal (i.e the measured reflected power is minimal), if the frequency at
which this happens coincides with the required frequency then we stop
adjusting $C_s$ and $C_p$ and heighten the current flowing to the antenna using $C_a$.
\begin{figure}[h]
\centering
\begin{circuitikz}[european] \draw
  (0,0) to [short, *-] (4,0)
  (4,0) to [vC, l=$C_p$] (4,4)
  (4,0)  to [short, *-] (6,0)
  node[ground] (6,0)
  to [short, *-] (8,0)
  to [short, *-] (12,0)
  (0,0) node[below]{$B$} to [open, v^>=$P_s$]  (0,4) 
  (0,4) node[above]{$A$} to [short, *- ,i=$i_{in}$] (1,4) 
  (4,4) to [short, *-] (0,4)
  (4,4) node[above]{$C$} to [vC, l=$C_{s}$] (8,4) 
  (12,0) to [vC, l=$C_{a}$] (12,4) 
  (8,4) node[above]{$E$} to [R, l=$Z_{Ant,s}$]  (12,4)
  (12,4) node[above]{$G$}
  (4,0) node[below]{$D$}
  (8,0) node[below]{$F$}
  (12,0) node[below]{$H$}
  (8,0) to [R, l=$Z_{Ant,p}$]  (8,4);
  %\node at (1.5,2) (loop1) {$\ \ \qquad1$}
  %\draw [->] (loop1.south)arc(-160:160:0.5);
  %\node at (5.5,2) (loop2) {$\ \ \qquad2$}
  %\draw [->] (loop2.south)arc(-160:160:0.5);
  %\node at (9.5,2) (loop3) {$\ \ \qquad3$}
  %\draw [->] (loop3.south)arc(-160:160:0.5);
\end{circuitikz}
\caption{
The Power input is marked $P_s$, the antenna can be described as having a
parallel impedence $Z_{Ant,p}$ and a series impedence $Z_{Ant,s}$. The power
input is assumed to have an entry impedence of $Z_0=50\Omega$ and to match the
circuit we can play around with the variable parallel matching $C_p$, the
series matching $C_s$ and the pre-matching $C_a$ capacitors}
\label{fig:Circuit}
\end{figure}

\section{Previous work}
Work has been done on this previously by F. Fasseur\cite{Saussez}, which lays
out a good exposition on the system, how changing the various capacitors affects the system 
and which capacitance values of $C_a$ result in a matchable system.\\

K. Elinck continued on this work by implementing a neural network and training it on simulated data\cite{Gen2019},
we question wether the simulated training corresponds well with the experiment and if it's even needed to make the algorithm
so complex (as e.g TEXTOR's circuit worked fine\cite{DURODIE1993477} which was just a simple linear algorithm) and as such 
we won't go into detail on his work.
\section{system}
To understand how to, in practice, implement an algorithm, the interconnectedness of the ICRH system needs to be discussed:
A computer which we'll henceforth call pc1 is connected to an Arduino which can modify the capacitance of the capacitors
with steppermotors. pc1 is also connected to the ICRH amplifier
making it possible to select the power.

The DAQ, however, is a second computer: pc2. We can thus set the various
parameters on pc1 but only see what they imply on pc2, making it necessary to
transfer data between the two, which makes the automatic matching a little more
complex.
