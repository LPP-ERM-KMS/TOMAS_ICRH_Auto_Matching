\section{How to proceed}
There are two ways in which We can proceed:
\begin{itemize}
	\item Either we continue on the work of F.Fasseur which means
		choosing a frequency, measuring various voltages and currents in the line,
		calculating "error values" and adjusting the capacitors using these error values.
	\item Or we can automate the previous manual matching procedure
\end{itemize}
We'll implement the second option as it might be the easiest, henceforth called \textit{frequency sweep matching} as opposed to the first option which we'll call \textit{error matching}.

\section{Appendix: Technical}
The ICRH antenna has a matching system consisting of a parallel capacitor, a
series capacitor and a pre-matching capacitor. These capacitors are variable
and their capacitance can be manipulated by moving the plates, step motors were
attached to move these, each connected to an arduino. The arduino has them
labeled 'S' for the series capacitor in the L-box, 'P' for the parallel
capacitor in the L-box and 'A' for the pre-matching capacitor and the program
can be given a command of the form X x Y Y Z z with X,Y,Z in [A,P,S,X,Y,Z]
(X,Y,Z are, in order, the sample manipulator, the Horizontal triple probe and
the vertical triple probe) and x,y,z in [1,100] (position of plates in motor
steps). Their capacitances have ranges\\ 
$C_a \in \{35-1000pF\}$\\
$C_s \in \{25-1000pF\}$\\
$C_p \in \{7-1000pF\}$\\
. A circuit of
the ICRH antenna with it's matching boxes is shown below. 

\secion{Matching Code}
\section{Python code}
\begin{lstlisting}[language=Python]
def MatchICRH(self):
	####################################
	#            constants             #
	####################################

	#length from amplifier to voltmeters 1-3
	l1 = ?
	l2 = ?
	l3 = ?
	#speed of light in m/s
	c = 299792458
	# not matched yet
	matched = False

	if self.matchICRH_entr == "":
	    print("frequency must be specified")
	    return

	####################################
	# move C_a to the correct position #
	####################################
	f = self.matchICRH_entr.get()
	C_a = -3.3094*e-6*f**5 5.17212e-4*f**4 - 5.10998e-2*f**3 + 3.5211*f**2 - 132.074*f + 2000.87 #calculate required capacitance
	Stepfactor = 10 #amount of steps in a full revolution
	StepToMoveTo = round(Stepfactor*(0.04781*C_a - 3.68067)) -  self.minPos[0]#The 'A' limits are from minPos to maxPos whilst the original c_a to steps assumes starting at 0
	cmd = ""
	cmd += "A " + moveAto + " "
	if cmd == "":
	    print("Some error occured whilst trying to move the capacitor")
	    return
	# Communicate the desired position to Arduino
	print(cmd)
	self.arduino.write(cmd.encode())
	time.sleep(2) #buffer, 
	# Retrieve communication from Arduino
	if "Error" in newPos.decode():
	    print(newPos.decode())
	    # After the error, Arduino will communicate the new positions
	print("Theposition of C_a is:")
	print(newPos.decode())
	self.f.write(newPos.decode().strip()+"\n")
	# Update the information on the GUI
	posStrs = newPos.decode().split(" ")
	for i in range(0, len(posStrs)):
	    if posStrs[i] == "A": #check to make sure, but probably not needed
		self.posA_lbl.config(text="A: " + posStrs[i + 1].strip())

	#######################################
\end{lslisting}
